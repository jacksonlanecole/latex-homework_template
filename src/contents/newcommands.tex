% --------------------------------------------------------------------------- %
% Essential newcommands
% --------------------------------------------------------------------------- %
% \sn: scientific notation
\newcommand{\sn}[2]{#1\times10^{#2}}
% \functionof: \functionof{f}{t, x} = f(t, x)
\newcommand{\functionof}[2]{#1\left(#2\right)}
% \unit: makes a unit vector symbol out of the argument
\newcommand{\unit}[1]{\,\hat{\bm{#1}}}
% \vect: makes a bold vector
\newcommand{\vect}[1]{\mathbf{#1}}
% \units: this thing is great for writing units inside of math mode
\newcommand{\units}[1]{\,\text{#1}}
% \fracunits: this thing is great for writing units that must be in fractions
%     - The first argument is either \frac or \dfrac, depending on which
%     fractional style you want. I assume you could also use other fractional
%     styles
\newcommand{\fracunits}[3]{#1{\units{#2}}{\units{#3}}}

% The following commands are easier than writing \left( and \right)
% Instead, just write \paren{arg}, \brackets{arg}, etc.
\newcommand{\paren}[1]{\left(#1\right)}
\newcommand{\brackets}[1]{\left[#1\right]}
\newcommand{\anglers}[1]{\langle #1 \rangle}

% The following commands add parentheses to the original commands, or just make
% them work like I think they should work.
\newcommand{\abs}[1]{\left|#1\right|}
\newcommand{\factorial}[1]{#1!}
\newcommand{\factorialp}[1]{\factorial{\paren{#1}}}
\newcommand{\lnp}[1]{\ln{\paren{#1}}}
\newcommand{\logten}[1]{\log_{10}\paren{#1}}
\newcommand{\sinp}[1]{\sin{\paren{#1}}}
\newcommand{\cosp}[1]{\cos{\paren{#1}}}
\newcommand{\tanp}[1]{\tan{\paren{#1}}}
\newcommand{\expp}[1]{\exp\paren{#1}}
\newcommand{\integral}[2]{\int_{#1}^{#2}}
\newcommand{\summ}[2]{\sum_{#1}^{#2}}
\newcommand{\deriv}[2]{\dfrac{d #1}{d #2}}

\newcommand{\degrees}[1]{\,\degree\units{#1}}

% some physics specific symbols (I need to find better symbols)
\newcommand{\lagr}{\mathcal{L}}
\newcommand{\ham}{\mathcal{H}}

% Some partial derivative commands that were useful during E&M
\newcommand{\partialwoa}[1]{\dfrac{\partial}{\partial #1}}
\newcommand{\partiald}[2]{\dfrac{\partial}{\partial #2}\brackets{#1}}
\newcommand{\partialdd}[2]{\dfrac{\partial^2}{\partial^2 #2}\brackets{#1}}

% I think this is a triple integral over a whole sphere
\newcommand{\tsi}[1]{\int_{\phi=0}^{2\pi}\int_{\theta=0}^{\pi}\int_{r=0}^{#1}}
